% This is an LPSC Abstract template for LaTeX 2e that is based off of the
% LaTeX article document class.

% $Id: lpsc_abstract.tex 22 2016-01-08 00:28:19Z rbeyer $

% Copyright (C) 2005,2007 Ross A. Beyer
% Copyright (C) 2008 Ross A. Beyer and Moses P. Milazzo

% This work is licensed under the Creative Commons
% Attribution-Noncommercial-Share Alike 3.0 License. To view a copy
% of this license, visit http://creativecommons.org/licenses/by-nc-sa/3.0/;
% or, (b) send a letter to Creative Commons, 171 2nd Street, Suite
% 300, San Francisco, California, 94105, USA.

% Use the LaTeX article class.  Pass any options you like to article, except
% twocolumn.  We take care of that in the lpscabs pacakage below.
\documentclass[twoside]{article} 

\usepackage[margin=1in]{geometry}	% Sets margins cleanly
\usepackage{times}					% Nice fonts, optional
\usepackage{url}                	% Needed for wrapping long URLs


\usepackage{graphicx}			% Nice graphics, optional

% To use Biber:
\usepackage[backend=biber,style=numeric,natbib=true,sorting=none]{biblatex}
\addbibresource{bibliography.bib}
\defbibheading{bibliography}[\refname]{\paragraph{#1:}}
\defbibenvironment{bibliography}{\inparaenum[ {[}\printfield{labelnumber}{]}]}{\endinparaenum}{\textbf{\item}}

% To use Bibtex:
% \usepackage[numbers]{natbib}	% Nice references, optional
% % These next three commands are relevant to natbib.  If you don't use
% % natbib for your references, you should get rid of these, as well.
% \renewcommand{\bibfont}{\small}	% Change the font size of the bibliography
% \setlength{\bibsep}{0pt}		% Remove the spacing between bib entries
% \renewcommand{\bibsection}{\subsubsection*{\refname}}
% 								% Make the References heading smaller.



% To compress the section titles even more, use the titlesec package.
\usepackage[small,compact]{titlesec}

\usepackage{paralist}			% For compressing bibliography into a paragraph

% \usepackage{balance}			% For balancing columns on the page.
								

\usepackage{lpscabs}			% This is to take care of some LPSC abstract
								% document-specific things and set sizes.

\usepackage[pdftex,colorlinks=true,urlcolor=blue,citecolor=black,linkcolor=black]{hyperref}
								% Puts actual hyperlinks in your PDF, optional.

\begin{document}

% The \titlearea command takes two arguments in curly braces {}.  The first
% will be used as the title, and the second as the author info.

\titlearea{Planetary Software Organization}{
		Andrew Annex$^{1}$,
		Michael Aye$^{2}$,
		Ross A. Beyer$^{3,4}$,
		Jay Laura$^{5}$,
		Jesse Mapel$^{5}$,
		Victor Silva$^{6}$,
		and Summer Stapleton$^{5}$.
		$^{1}$Johns Hopkins University,
		$^{2}$University of Colorado at Boulder,
		$^{3}$SETI Institute,
		$^{4}$NASA Ames Research Center,
		$^{5}$USGS Astrogeology Science Center,
		and $^{6}$Arizona State University.}

% Here is an example with just one author:
%
%\titlearea{Using \LaTeX\ to write an LPSC Abstract.}{}

% Here is an example with more than one author from more than one place:
%
% \titlearea{Using \LaTeX\ to write an LPSC Abstract.}{Ross A. Beyer$^{1,2}$, John Q. Public$^{a}$, and Jane Doe$^{\dag}$, $^1$Carl Sagan Center at the SETI Institute, $^2$NASA Ames Research Center, MS 245-3, Moffett Field, CA, USA (Ross.A.Beyer@nasa.gov), $^{a}$123 Sesame Street, New York NY, USA, $^{\dag}$890 Somewhere, ID, USA}

% Uncomment \usepackage{balance} above to use this:
% \balance

% FYI PSIDA abstracts are one page only!

The Planetary Software Organization promotes open source software
in the planetary sciences by helping software creators and maintainers
foster an open source community. More information can be found at
its web page: \url{https://github.com/planetarysoftware/TSC}.


The Planetary Software Organization helps software creators grow
and organize their community by providing a set of ready-to-use
community guidelines, standards, and governance documents. The
Planetary Software Organization also helps software maintainers
manage the growth and evolution of their open source software by
providing mentorship and suggestions.

The Technical Steering Committee (TSC) is the governing body of the
Planetary Software Organization, as described in its Charter. The TSC
was formed in 2018 and was modeled after the nodejs TSC. The nodejs
TSC charter was utilized as a template because it is a well developed 
framework for software governance for a large organization. 

The TSC primarily provides responsive input and organizational help
to its Top-Level Projects and Working Groups, but is also available
for consultation by other software projects or endeavors.


% Uncomment \usepackage{paralist} above to use the inparaenum environment.
%\begin{flushleft}
%\begin{inparaenum}

% To use Biber:
\printbibliography

% To use Bibtex:
%\bibliography{bibliography}
%\bibliographystyle{unsrtnat}

%% Your References section ridiculously long?  
%% Maybe try the unsrtetal .bst file included with this template, although
%% I haven't figured out how to make \citet commands work with it.
%% \bibliographystyle{unsrtetal}
%\end{inparaenum}
%\end{flushleft}

\end{document}

